\documentclass{beamer}
\usetheme{Berlin}
\usepackage[utf8]{inputenc}
\usepackage[T1]{fontenc}
\usepackage[utf8]{vietnam}
\usepackage{lmodern}
\usepackage[]{babel}
\usepackage{listings}
\usepackage[vietnamese]{babel}

\lstset{
	language=C++,
	basicstyle=\ttfamily\scriptsize,
	keywordstyle=\color{blue}\ttfamily,
	stringstyle=\color{red}\ttfamily,
	commentstyle=\color{green!50!black}\ttfamily,
	morecomment=[l][\color{magenta}]{\#},
	backgroundcolor=\color{white},
	frame=tb,
	showstringspaces=false,
	columns=fullflexible,
	keepspaces=true,
	xleftmargin=10pt,
	aboveskip=5pt,
	belowskip=5pt,
	numbers=left,
	numbersep=5pt,
	numberstyle=\tiny\color{gray}
}

\begin{document}
	
	\author{Nguyễn Đức Hùng}
	\title{Đường đi ngắn nhất}
	%\subtitle{}
	%\logo{}
	%\institute{}
	%\date{}
	%\subject{}
	%\setbeamercovered{transparent}
	%\setbeamertemplate{navigation symbols}{}
	\begin{frame}[fragile]
		\maketitle
	\end{frame}
	
	
	\begin{frame}[fragile]{Code Presentation}
		
		\begin{lstlisting}
			#include <iostream>
			#include <math.h>
			
			using namespace std;
			
			int main() {
				long long n, m;
				cin >> n >> m;
				
				if(n % 2 == 0) {
					long long x = (n / 2);
					long long y = n;
					
					long long k = (x + m - 1) / m * m;
					cout << ((k <= y)? k : -1);
				}
				else if(n % 2 != 0) {
					long long x = (n + 1) / 2;
					long long y = n;
					
					long long k = (x + m - 1) / m * m;
					cout << ((k <= y)? k : -1);
				}
				
				return 0;
			}
		\end{lstlisting}
		
		\begin{itemize}
			\item This is a simple C++ code that reads in two long long integers $n$ and $m$ from standard input and outputs a long long integer to standard output.
			\item The code uses the if-else conditional statement to compute the value of the output integer based on the input integers.
			\item Note that the code uses the cmath library by including $<$math.h$>$ header file.
		\end{itemize}
		
	\end{frame}
	
	\begin{frame}{Giải thích}
		\begin{itemize}
			\item Dòng đầu tiên và dòng thứ hai khai báo các thư viện tiêu chuẩn của C++.
			\item Dòng thứ tư khai báo sử dụng không gian tên \texttt{std} để không phải ghi \texttt{std::} trước các hàm.
			\item Dòng thứ sáu khai báo hàm \texttt{main} là hàm chính của chương trình.
			\item Dòng thứ bảy và dòng thứ tám khai báo hai biến số nguyên dài (\texttt{long long}) \texttt{n} và \texttt{m} và đọc giá trị của chúng từ bàn phím.
			\item Dòng
		\end{itemize}
		
	\end{frame}
	
	\begin{frame}{Lời cảm ơn!}
		
		Cảm ơn các bạn đã quan tâm và theo dõi! Lần đầu code và soạn theo kiểu này nên vẫn còn sai sót, mong mọi người bỏ qua
		
	\end{frame}
\end{document}
